\section{Saját jelölések}
\begin{pl}
	Példa
\end{pl}
\begin{jel}
	Az $\hz{i\in\mathbb{N}|1\leq i\leq n}$ halmazt \vjel{$\underline{n}$} jelöli.
\end{jel}
\begin{jel}
	A szokásos halmazokon értelmezett kivonást a $-$ jel fogja jelölni
	$\backslash$ helyett
\end{jel}
\begin{pl}
	A nemnulla valós számok halmaza: $\R-\hz{0}$
\end{pl}

\section{Vektorterek}
\begin{ff}
	Ha $(X,\leq)$ \deff{parciálisan rendezett halmaz}, ha $\leq$ parciális
	rendezés.
\end{ff}
\begin{ff}
	Ha $(X,\leq)$ parciálisan rendezett halmaz, ekkor $a\in X$
	\deff{maximális}, ha $\forall b\in X$, $a\leq b$ esetén $a=b$.
\end{ff}
\begin{ff}
	Ha $(X,\leq)$ parciálisan rendezett halmaz, ekkor $Y\subset X$ \deff{lánc},
	ha $(Y,\leq)$ teljesen rendezett halmaz.
\end{ff}

\section{Alterek összege}
\begin{jel}
	\vjel{$V_K$} jelöli a $V$ vektorteret $K$ test felett, de következetesség miatt
	mindig megjegyezzük, hogy "$V_K$ vektortér", ami még mindig rövidebb,
	mint a hosszú "$V$ $K$ test feletti vektortér".
\end{jel}

\begin{ff}
	Ha $V_K$ vektortér és $V_1,\ldots,V_n\leq V$, akkor ezen alterek
	\deff{összege}:
	\begin{equation*}
		V_1+\ldots+V_n=\hz{v_1+\ldots +v_n|\,v_i\in V_i,\,i\in\underline{n}}
	\end{equation*}
\end{ff}
\begin{ff}
	Ha $V_K$ vektortér és $V_1,\ldots,V_n\leq V$, akkor ezen alterek
	\deff{direkt összege}, olyan alterek összege, aminek minden eleme
	egyértelműen áll elő $v_1+\ldots+v_n$ ($\forall v_i\in
	V_i, i\in\underline{n}$) alakban.
\end{ff}
\begin{ff}
	$V_k$ vektortér, $U,W\leq V$, $V=U\oplus W$, $\pi :V\to V$ lineáris
	leképezés, $u\in U$, $w\in W$, hogy $v=u+w \in V$ esetén $\pi(v)=u$.
	Ekkor $\pi$-t az \deff{$U$ altérre való $W$ irányú
	vetítésnek nevezzük.}
\end{ff}

\section{Konjugált mátrixok}
\begin{ff}
	$A,B\in K^{n\times n}$ $A$ és $B$ \deff{konjugáltak}
	(hasonlók), ha létezik egy olyan $X\in K^{n\times n}$, melyre
	$B=X^{-1}AX$.
\end{ff}
\begin{ff}
	$V_K$ vektortér, $f:V\to V$ linráris leképezés a $V$
	\deff{endomorfizmusa}.
\end{ff}
\begin{jel}
	\vjel{$\End_KV$}
\end{jel}
\begin{ff}
	$V_K$ vektortér, $f\in \End_KV$, $t\in K$:
	\begin{enumerate}
		\item $t\in K$ \deff{sajátértéke} $f$-nek, ha létezik
			egy olyan nemnulla $V$-beli vektor, amire $f(v)=tv$,
		\item $t\in K$ sajátérték, ekkor a $V$-beli $v$ vektor a
			\deff{$t$-hez tartozó sajátvektor}, ha $f(v)=tv$.
		\item az $f$ összes sajátértékének halmaza $f$
			\deff{spektruma}.
	\end{enumerate}
\end{ff}
\begin{ff}
	Ha $n\ge 1$, $A\in K^{n\times n}$, ekkor $A$ \deff{mátrix
	sajátértékei, sajátvektorai, spektruma} az $f_A:K^n\to K^n$, $x\mapsto Ax$
	endomorfizmus sajátértékei, sajátvektorai és spektruma.
\end{ff}
\begin{ff}
	$V_K$ vektortér, $f\in \End_KV$,az $S_t$ a $t$-hez tartozó sajátvektorokat tartalmazó
	halmaz, akkor $S\cup {0_V}$ az $f$ endomorfizmus $t$-hez tartozó
	\deff{sajátaltere}.
\end{ff}
\begin{jel}
	\vjel{$\Eig_{f,t}$}
\end{jel}
\begin{ff}
	$V_K$ vektortér $f\in \End_KV$ , $t$ változó, akkor $\vdet(f-tI)$ 
	polinom az $f$ \deff{karakterisztikus polinomja}.
\end{ff}
\begin{jel}
	\vjel{$\vchar_f(t)$}
\end{jel}
\begin{ff}
	$V_K$ vektortér, $f\in \End_KV$, $t\in K$, ekkor $\vdim(\Eig_{f,t})$ a $t$ 
	sajátártákánek \deff{geometriai multiplicitása}.
\end{ff}
\begin{ff}
	$V_K$ vektortér, $t_0$ az $f\in\End_VK$ sajátértéke, ekkor $t_0$ \deff{algebrai
	multiplicitása} $k$, ha $t_0$ pntosan $k$-szoros gyöke $\vchar_f(t)$-nek.
\end{ff}
\begin{ff}
	$V_K$ vektortér, ekkor $f\in\End_KV$ \deff{diagonalizálható}, ha létezik egy
	$\mathcal{B}$ bázis, amiben $[f]_{\mathcal{B}}$ diagonális.
\end{ff}

\section{Mátrixok sajátfelbontása}
\begin{ff}
	$A\in K^{n\times n}$, $y\in K^n-{0}$, akkor az $y^T$ sorvektor az $A$
	\deff{baloldali sajátvektora}, ha $y^TA=\lambda y^T$ 
\end{ff}
\begin{ff}
	$A\in K\nn$ diagonilazálható mátrix \deff{sajátfelbontása} $PDP^{-1}$, ahol $P$
	$i$-edik oszlopa az $A$ mátrixhoz tartozó $t_i$-edik egyik sajátvektora
	(jelölje ezt most $\underline{x}_i$), $P^{-1}$ $i$-edik sora az $A$ mátrix
	$t_i$-hez tartozó egyik baloldali sajátvektora (jelölje most ezt
	$\underline{y}_i)$. Ekkor:
	\begin{equation*}
		\sum_{i=1}^nt_i\underline{x}_i\underline{y}_i
	\end{equation*}
	a \deff{sajátfelbontás diadikus alakja}.
\end{ff}

\section{Spektrálfelbontás}
\begin{ff}
	Ha $A\in K\nn$-nak létezik sajátfelbontása és $P_i$ a $\Eig_{A,t(i)}$-re való
	vetítés mátrixa (a vetítés iránya a többi sajátaltér direkt összege),
	akkor $A$ \deff{spektrálfelbontása}:
	\begin{equation*}
		A=t_1P_1+\ldots+t_kP_k
	\end{equation*}
\end{ff}

\section{Bilineáris leképezések}
Dián vannak itt dolgok, amit nem akarok leírni.

Dia (deffiníció)tartalma:
\begin{enumerate}
	\item bilineáris leképezés
	\item szimmetrikus bilineáris leképezés
	\item Gram-mátrix
	\item baloldali, jobboldali mag, reguláris
	\item ortogonalitás bilineáris leképezésre nézve
	\item ortogonális kiegészítő
	\item Tehetetlenségi Tétel
\end{enumerate}

\section{Euclides-terek}
\begin{ff}
	$V_{\mathbb{R}}$, $\beta:V\times V\to \mathbb{R}$ bilineáris leképezés,
	ekkor $\beta$ \deff{pozitív definit}, ha szimmetrikus és minden V-beli
	$v$ vektorra $\beta(v,v)\ge 0$, továbbá $\beta(v,v)=0$ pontosan akkor,
	ha $v=0$. Másik elnevezés a \deff{skalárszorzat}.
\end{ff}

\begin{ff}
	~\begin{enumerate}
		\item $V_{\mathbb{R}}$, $\beta:=\scp{.,.}:V^2\to \mathbb{R}$
			skalárszorzat (másnéven belsőszorzat), ekkor a $(V,\scp{.,.})$ pár
			\deff{Euclides-tér} (inner product space).
		\item $v\in V$, akkor $\vnorm{v}=\sqrt{\scp{v,v}}$ a $v$ \deff{normája}
		\item $v,w\in V$, akkor $d(v,w)=||v-w||$ a $v$ és a $w$ \deff{távolsága}
	\end{enumerate}
\end{ff}
\begin{jel}
	\vjel{$(V,\scp{.,.})_E$} jelöli a $V_\mathbb{R}$ vektorteret, amiben $\scp{.,.}$ 
	vektorszorzat. Ha emellett $V$ még véges dimenziós is, akkor ezt
	\vjel{$(V,\scp{.,.})_{E,\fin}$} jelöli.
\end{jel}
\begin{ff}
	$\euclsp$ ekkor ha $v,w\in V$, akkor a \deff{szögük}:
	\begin{equation*}
		\arccos\frac{\scp{v,w}}{\vnorm{v}\cdot\vnorm{w}}
	\end{equation*}
\end{ff}
\begin{ff}
	$\euclsp$ és $v,w\in V$, akkor $v\perp_{\scp{.,.}}w$, ha $v$ és $w$ szöge $\frac{\pi}{2}$.
\end{ff}

\section{Ortogonális bázisok}
\begin{ff}
	$\euclsp$ $S\subset V$ \deff{ortogonális} részhalmaz, ha minden $v,w\in S$ 
	esetén $v\perp_{\scp{.,.}}w$, valamint $S$ \deff{ortonormált}, ha
	ortogonális és minden $S$-beli $v$-re $\vnorm{v}=1$, és $S$
	\deff{ortonormált bázis}, ha ortonormált és bázis.
\end{ff}
\begin{ff}
	$A\in\R^{\nn}$ \deff{pozitív definit mátrix}, ha $\beta(x,y)=x^TAy$
	pozitív definit pontosan akkor, ha $A$ szimmetrikus és $x^Tx>0$ minden
	nemnulla $\R^n-{0}$-beli vektorra.
\end{ff}

\section{Ortogonális kiegészítő}

\section{Adjungált}
\begin{ff}
	$(V_1,\scp{.,.}_1)_E$, $(V_2,\scp{.,.}_2)_E$, $f:V_1\to V_2$ lineáris leképezés
	$f^*:V_2\to V_1$ lineáris függvényt az $f$ \deff{adjungáltjának} nevezzük, ha
	minden $v$ $V_1$-beli,$w$ $V_2$-beli vektorokra:
	\begin{equation*}
		\scp{f(v),w}_2=\scp{v,f^*(w)}_1.
	\end{equation*}
\end{ff}
\begin{ff}
	$\euclsp$ $f\in\End_\R V$, ekkor $f$ \deff{önadjungált}, ha $f^*=f$.
\end{ff}
\begin{ff}
	$X\in\R\nn$ \deff{ortogonális}, ha $X^TX=I$ (azaz $X^T=X^{-1}$).
\end{ff}

\section{Ortogonális transzformációk}
\begin{ff}
	$\euclsp$, $f\in\End_\R V$, ekkor $f$ \deff{ortogonális transzformácó}, ha
	bijektív (izomorfizmus) és minden $V$-beli $v$ és $w$ esetén:
	\begin{equation*}
		\scp{v,w}=\scp{f(v),f(w)}.
	\end{equation*}
\end{ff}


\begin{ff}
	A $(G,\cdot)$ páros \deff{csoport}, ha $G\neq\emptyset$.
\end{ff}
\begin{ff}
	A $(G,\cdot)$ páros \deff{Abel-csoport}, ha $(G,\cdot)$ csoport és $\cdot$ 
	kommutatív.
\end{ff}


\begin{jel}
	\vjel{$\vO_{\scp{.,.}}$} (vagy \vjel{$\vO_{\scp{}}$}).
\end{jel}

\begin{ff}
	Ha $\fineuclsp$, $V=\R\nn A\in V$ $\scp{.,.}$-ortogonális leképezés mátrixa, ekkor az
	ilyen leképezések csoportja a mátrixszorzásra nézve az \deff{n-edrendű
	ortogonális csoport}.
\end{ff}

\begin{jel}
	\vjel{$\vO_{n,\scp{}}(\R)$}
\end{jel}

\begin{ff}
	$\fineuclsp$, $f\in\vO(V)$, ekkor az olyan $f$-ek melyekre $det(f)=1$ a
	\deff{speciális ortogonális csoportot alkotnak}.
\end{ff}

\begin{jel}
	\vjel{$\SO(V)$}
\end{jel}



\begin{ff}
	Ha $A:\R^2\to \R^2$ mátrix:
	\begin{equation*}
		A=	
		\begin{bmatrix}
			\cos t & -\sin t\\
			\sin t & \cos t
		\end{bmatrix}
	\end{equation*}
	alakú, akkor \deff{kétdimenziós forgásmátrix}.
\end{ff}

\begin{jel}
	\vjel{$R_t$}
\end{jel}


\begin{kov}
	Két síkra való tükrözés tengely körüli forgatás.
\end{kov}


\section{Szemiortogonális mátrixok}
\begin{ff}
	$A\in\R\mn$ \deff{szemiortogonális}, ha az oszlopok vagy sorok ortonormált
	bázist (vagy csak rendszert?) alkotnak. Tehát $A^TA=I_n$, ha az oszlopok
	alkotnak, $AA^T=I_m$, ha a sorok alkotnak ortonormált rendszert.
\end{ff}
\begin{ff}
	A teljes oszloprangú ($\rank{A}=n$) $A\in\R\mn$ \deff{QR-felbontása} $QR=A$, ha
	$Q\in\R\mn$ szemiortogonális, $R\in\R\nn$ pedig felsőháromszögmátrix, ahol a
	diagonális elemek nemnegatívok.
\end{ff}
\begin{ff}
	$\fineuclsp$, $\beta:V\times V\to \R$, szimmetrikus, bilineáris, akkor
$\beta:$	
	\begin{enumerate}
		\item \deff{pozitív definit}, ha $\forall v\in V:$ $\beta(v,v)\ge 0$ és
			$\beta(v,v)=0\leftrightarrow v=0$,
		\item \deff{pozitív semidefinit}, ha $\forall v\in V:$ $\beta(v,v)\ge 0$,
		\item \deff{negatív definit}, ha $\forall v\in V:$ $\beta(v,v)\leq 0$ és
			$\beta(v,v)=0\leftrightarrow v=0$,
		\item \deff{negatív semidefinit}, ha $\forall v\in V:$ $\beta(v,v)\leq 0$,
		\item \deff{indefinit}, ha $\exists v,w:$ $\beta(v,v)>0$, $\beta(w,w)<0$.
	\end{enumerate}
\end{ff}
