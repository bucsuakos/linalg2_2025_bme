%theorem declarations
\declaretheoremstyle[
	spaceabove=6pt,
	headfont=\normalfont\bfseries,
	bodyfont=\normalfont,
	headformat=swapnumber,
	headpunct = {}
]{tetel}

\declaretheoremstyle[
	spaceabove=6pt,
	headfont=\normalfont\bfseries,
	bodyfont=\normalfont,
	headformat=swapnumber,
	headpunct = {}
]{deffinicio}

\declaretheoremstyle[
	spaceabove=6pt,
	headfont=\normalfont\bfseries,
	bodyfont=\normalfont,
	headformat=swapnumber,
	headpunct = {}
]{jeloles}

\declaretheoremstyle[
	spaceabove=6pt,
	headfont=\normalfont\bfseries,
	bodyfont=\normalfont,
	headformat=swapnumber,
	headpunct = :
]{bizonyitas}

\declaretheorem[style=jeloles,shaded={rulecolor=jelruler, rulewidth=1pt,
bgcolor=jelbackround},name=Jelölés, numbered=no]{jel}

\declaretheorem[name=Dud]{dud}

\declaretheorem[sibling=dud,shaded={rulecolor=tetelruler,rulewidth=1pt,
bgcolor=tetelbackground},name=Tétel]{tet}

\declaretheorem[sibling=dud,shaded={rulecolor=tetelruler,rulewidth=1pt,
bgcolor=tetelbackground},name=Állítás]{all}

\declaretheorem[sibling=dud,shaded={rulecolor=white,
rulewidth=1pt,bgcolor=tetelbackground},name=Lemma]{mm}

\declaretheorem[sibling=dud,name=Következmény]{kov}

\declaretheorem[sibling=dud,style=tetel,shaded={rulecolor=white, rulewidth=1pt,
bgcolor=deffbackground},name=Deffiníció]{ff}

\declaretheorem[style=deffinicio,sibling=dud,name=Axióma]{ax}

\declaretheorem[style=bizonyitas,name=Bizonyítás]{biz}

\newenvironment{pl}
	{\begin{center}

	\begin{tabular}{|p{0.9\textwidth}}
	\hline\\
	}
	{ 
	\\\\\hline
	\end{tabular} 
	\end{center}
	}

\newenvironment{megj}
	{\begin{center}

	\begin{tabular}{||p{0.9\textwidth}}
	}
	{ 
	\end{tabular} 
	\end{center}
	}


