%common shorthands:
	%all - állítás
	%mm - lemma
	%ff - deffiníció
	%ax - axióma
	%tet - tétel
	%jel - jelölés
	%biz - bizonyítás

%theorem declarations
%dud comes here because it is the sibling of all numbered declared theorems.
%There is no seperate Dud theorem, it doesn't have a style
\declaretheorem[name=Dud]{dud}

\declaretheoremstyle[
	spaceabove=6pt,
	headfont=\normalfont\bfseries,
	bodyfont=\normalfont,
	headformat=swapnumber,
	headpunct = {}
]{tetel}
\declaretheorem[style=tetel,sibling=dud,shaded={rulecolor=tetelruler,rulewidth=0pt,
bgcolor=tetelbackground},name=Tétel]{tet}

\declaretheoremstyle[
	spaceabove=6pt,
	headfont=\normalfont\bfseries,
	bodyfont=\normalfont,
	headformat=swapnumber,
	headpunct = {}
]{allitas}
\declaretheorem[style=allitas,sibling=dud,shaded={rulecolor=allruler,rulewidth=0pt,
bgcolor=allbackground},name=Állítás]{all}

\declaretheoremstyle[
	spaceabove=6pt,
	headfont=\normalfont\bfseries,
	bodyfont=\normalfont,
	headformat=swapnumber,
	headpunct = {}
]{lemma}
\declaretheorem[style=lemma,sibling=dud,shaded={rulecolor=mmruler,rulewidth=0pt,
bgcolor=mmbackground},name=Lemma]{mm}

\declaretheoremstyle[
	spaceabove=6pt,
	headfont=\normalfont\bfseries,
	bodyfont=\normalfont,
	headformat=swapnumber,
	headpunct = {}
]{kovetkeztetes}
\declaretheorem[style=kovetkeztetes,sibling=dud,shaded={rulecolor=kovruler,rulewidth=0pt,
bgcolor=kovbackground},name=Következmény]{kov}

\declaretheoremstyle[
	spaceabove=6pt,
	headfont=\normalfont\bfseries,
	bodyfont=\normalfont,
	headformat=swapnumber,
	headpunct = {}
]{deffinicio}
\declaretheorem[sibling=dud,style=tetel,shaded={rulecolor=deffruler, rulewidth=0pt,
bgcolor=deffbackground},name=Deffiníció]{ff}

\declaretheoremstyle[
	spaceabove=6pt,
	headfont=\normalfont\bfseries,
	bodyfont=\normalfont,
	headformat=swapnumber,
	headpunct = {}
]{axioma}
\declaretheorem[style=axioma,sibling=dud,shaded={rulecolor=axruler, rulewidth=0pt,
bgcolor=axbackground},name=Axióma]{ax}

%jel is not the sibling of dud (not numbered)
\declaretheoremstyle[
	spaceabove=6pt,
	headfont=\normalfont\bfseries,
	bodyfont=\normalfont,
	headformat=swapnumber,
	headpunct = {}
]{jeloles}
\declaretheorem[style=jeloles,shaded={rulecolor=jelruler, rulewidth=1pt,
bgcolor=jelbackround},name=Jelölés, numbered=no]{jel}

%biz is not the sibling of dud (not numbered)
\declaretheoremstyle[
	spaceabove=6pt,
	headfont=\normalfont\bfseries,
	bodyfont=\normalfont,
	headformat=swapnumber,
	headpunct = :
]{bizonyitas}
\declaretheorem[style=bizonyitas,shaded={rulecolor=bizruler, rulewidth=1pt,
bgcolor=bizbackground},name=Bizonyítás]{biz}


\newenvironment{pl}
	{\begin{center}
		\begin{tabular}{|p{0.9\textwidth}}
		}
		{ 
		\end{tabular} 
	\end{center}
	}

\newenvironment{megj}
	{\begin{center}
		\begin{tabular}{||p{0.9\textwidth}}
		}
		{ 
		\end{tabular} 
	\end{center}
	}
